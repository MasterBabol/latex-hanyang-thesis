
계엄을 선포한 때에는 대통령은 지체없이 국회에 통고하여야 한다. 국교는 인정되지 아니하며, 종교와 정치는 분리된다. 일반사면을 명하려면 국회의 동의를 얻어야 한다.

헌법재판소의 조직과 운영 기타 필요한 사항은 법률로 정한다. 공무원의 신분과 정치적 중립성은 법률이 정하는 바에 의하여 보장된다. 대통령의 임기가 만료되는 때에는 임기만료 70일 내지 40일전에 후임자를 선거한다.

헌법재판소는 법률에 저촉되지 아니하는 범위안에서 심판에 관한 절차, 내부규율과 사무처리에 관한 규칙을 제정할 수 있다. 국회의원은 국회에서 직무상 행한 발언과 표결에 관하여 국회외에서 책임을 지지 아니한다.

법관은 헌법과 법률에 의하여 그 양심에 따라 독립하여 심판한다. 국군의 조직과 편성은 법률로 정한다. 모든 국민은 양심의 자유를 가진다. 모든 국민은 법 앞에 평등하다. 누구든지 성별·종교 또는 사회적 신분에 의하여 정치적·경제적·사회적·문화적 생활의 모든 영역에 있어서 차별을 받지 아니한다.

형사피의자 또는 형사피고인으로서 구금되었던 자가 법률이 정하는 불기소처분을 받거나 무죄판결을 받은 때에는 법률이 정하는 바에 의하여 국가에 정당한 보상을 청구할 수 있다.